\documentclass[a4paper]{book}
\usepackage[times,inconsolata,hyper]{Rd}
\usepackage{makeidx}
\usepackage[utf8]{inputenc} % @SET ENCODING@
% \usepackage{graphicx} % @USE GRAPHICX@
\makeindex{}
\begin{document}
\chapter*{}
\begin{center}
{\textbf{\huge HCB}}
\par\bigskip{\large \today}
\end{center}
\begin{description}
\raggedright{}
\inputencoding{utf8}
\item[Title]\AsIs{Human Cultural Boundaries}
\item[Version]\AsIs{0.0.0}
\item[Description]\AsIs{Creates seed populations with phoneme inventories that can grow, migrate, and create offshoot populations.  Phoneme inventories mutate when populations establish a new territory.}
\item[License]\AsIs{What license it uses}
\item[Encoding]\AsIs{UTF-8}
\item[LazyData]\AsIs{true}
\item[RoxygenNote]\AsIs{6.1.1.9000}
\item[Imports]\AsIs{mc2d, randomcoloR, uuid, numbers, philentropy, ade4}
\end{description}
\Rdcontents{\R{} topics documented:}
\inputencoding{utf8}
\HeaderA{AddBeringStrait}{Add Bering Strait}{AddBeringStrait}
\keyword{Barriers}{AddBeringStrait}
%
\begin{Description}\relax
Removes connections at the FirstStep stage of the Local structure to create "barriers" between cells.  Bering Strait Barriers are designed to create structures similar to the Bering strait entering North America, Traveling to Central America, then opening up into South America.
\end{Description}
%
\begin{Usage}
\begin{verbatim}
AddBeringStrait(P, firstStep)
\end{verbatim}
\end{Usage}
%
\begin{Arguments}
\begin{ldescription}
\item[\code{P}] A list of parameters.

\item[\code{firstStep}] The local directions created by OneStepDirections().
\end{ldescription}
\end{Arguments}
\inputencoding{utf8}
\HeaderA{AddShift}{Add Shift Phoneme}{AddShift}
\keyword{Horizontal}{AddShift}
%
\begin{Description}\relax
Allows a language to either gain a new phoneme and shift and existing phoneme to match another population.
\end{Description}
%
\begin{Usage}
\begin{verbatim}
AddShift(P, language, languages, local, phonemeRelatedness, index)
\end{verbatim}
\end{Usage}
%
\begin{Arguments}
\begin{ldescription}
\item[\code{P}] A list of parameters.

\item[\code{language}] The target language to be modified if possible.

\item[\code{languages}] All languages

\item[\code{local}] Locality.

\item[\code{index}] The target territory whose language may change.

\item[\code{phonemeProbab}] The probability of gaining each phoneme in the population.
\end{ldescription}
\end{Arguments}
\inputencoding{utf8}
\HeaderA{AddSnakeBarriers}{Add Snake Barriers}{AddSnakeBarriers}
\keyword{Barriers}{AddSnakeBarriers}
%
\begin{Description}\relax
Removes connections at the FirstStep stage of the Local structure to create "barriers" between cells.  Snake Barriers are lines with length and spacing defined by the parameters.  The barriers jut out from the east and west walls, alternating east, west, east, west.  This creates a snaking zig-zag pattern, hence the name.
\end{Description}
%
\begin{Usage}
\begin{verbatim}
AddSnakeBarriers(P, firstStep)
\end{verbatim}
\end{Usage}
%
\begin{Arguments}
\begin{ldescription}
\item[\code{P}] A list of parameters.

\item[\code{firstStep}] The local directions created by OneStepDirections().
\end{ldescription}
\end{Arguments}
\inputencoding{utf8}
\HeaderA{CardinalDirections}{Cardinal Directions}{CardinalDirections}
\keyword{Directions}{CardinalDirections}
%
\begin{Description}\relax
Calculates the terrritory numbers of locations around a target territory (also used for phoneme relatedness in the same way).
\end{Description}
%
\begin{Usage}
\begin{verbatim}
CardinalDirections(target, R, start, round, South, North, East, West, SE,
  NE, SW, NW)
\end{verbatim}
\end{Usage}
%
\begin{Arguments}
\begin{ldescription}
\item[\code{target}] The territory around which to get local territoies.

\item[\code{R}] The number of rows.

\item[\code{start}] How much to offset numbers (for phoneme structures).

\item[\code{round}] Whether to get a "round" set of territories (N, S, E, W only) for phonemes or a square set of territories (includes diagonals) for distance.

\item[\code{SE}] Whether to get the southeasrern territory.

\item[\code{NE}] Whether to get the northestern territory.

\item[\code{SW}] Whether to get the southwestern territory.

\item[\code{NW}] Whether to get the northwestern territory.

\item[\code{south}] Whether to get the southern territory.

\item[\code{north}] Whether to get the northern territory.

\item[\code{east}] Whether to get the eastern territory.

\item[\code{west}] Whether to get the western territory.
\end{ldescription}
\end{Arguments}
\inputencoding{utf8}
\HeaderA{DefineParameters}{Define Parameters}{DefineParameters}
\keyword{SimParam}{DefineParameters}
%
\begin{Description}\relax
Creates a parameter data structure for running simulations.
\end{Description}
%
\begin{Usage}
\begin{verbatim}
DefineParameters(Rows = 40, Cols = 50, ChanceExpand = 0.8,
  PopulationStartIndex = c(1, 2), NumPopulationPhonemes = rep(NA,
  length(PopulationStartIndex)), UsePopSize = TRUE,
  IndividualsStEmSuEM = c(1000, 10, 20, NA), MutationRate = 15,
  PhonemeDitribution = c(12, 24, 133), Consonants = 750,
  Vowels = 100, MinConsonant = 6, MinVowel = 6,
  PhonemeProbabilityType = "RealMimic", GrowthRate = 5,
  Barriers = FALSE, BarrierLength = 30, BarrierBreaks = 4,
  MutationTypeChance = rep(1/5, 5), HorizontalRate = 0.1,
  Bias = TRUE, Steps = 1, HorizontalLocal = TRUE,
  NumberRandomHorizontal = 8, UpRoot = TRUE, Death = TRUE,
  Bering = FALSE, MigrationSimSteps = 300, HorizontalSimSteps = 400,
  Waves = FALSE, Seed = NA)
\end{verbatim}
\end{Usage}
%
\begin{Arguments}
\begin{ldescription}
\item[\code{Rows}] The number of rows in the world matrix.

\item[\code{Cols}] The number of columns in the world matrix.

\item[\code{ChanceExpand}] The chance that a population will either move or send off a group of individuals to found a new population.

\item[\code{PopulationStartIndex}] The position in the matrix where each seed population starts.  The number of seed populations is defined by the number of starting indicies.

\item[\code{NumPopulationPhonemes}] The number of phonemes in each starting population.  If set to NA, this is decided by sampling from a distribution with min, mode, and made on the values from the PhonemeDistribution arguement.

\item[\code{UsePopSize}] Whether to take into account the the population size (number of people) when making decisions about moving, immegrating, and phoneme loss/addition biases.

\item[\code{IndividualsStEmSuEM}] Four related parameters: 1) The number of individuals a seed population stats with, 2) the minumum number of individuals required to make a founder party to settle a new territory, 3) the minumum number of individuals that must stay behind when a founder party is sent off, 4) the maximum number of individuals allowed to be in one founder party.

\item[\code{MutationRate}] The rate at which phonemes mutate.  E.g., if MutationRate==0.1, each phoneme in a populatiosn phoneme inventory has a 10\% chance to mutate.

\item[\code{Consonants}] The number of possible consonants in existence.  Default based on real phoneme data.

\item[\code{Vowels}] The number of possible vowels in existence. Default based on real phoneme data.

\item[\code{MinConsonant}] The minumum number a Consonants that can be in a population's phoneme inventory. Default based on real phoneme data.

\item[\code{MinVowel}] The minumum number a vowels that can be in a population's phoneme inventory. Default based on real phoneme data.

\item[\code{PhonemeProbabilityType}] The method by which phoneme probabilities are established.

\item[\code{GrowthRate}] When an integer, the nunmber of individuals added to each population every time step.  When a fraction, the percent that a population increases each timestep.

\item[\code{Barriers}] Whether to create "snake barriers" that limit the direction of migration in the matrix.

\item[\code{BarrierLength}] The width of snake barriers.

\item[\code{BarrierBreaks}] The height of the space between snake barriers.

\item[\code{MutationTypeChance}] The chance that each mutation type occurs.  1) Add, 2) Lose, 3) Split, 4) Join, 5) Shift.

\item[\code{HorizontalRate}] The fraction of the population that attempts to modify its phoneme inventory every horizontal timestep.

\item[\code{Bias}] Whether to randomly bias mutations towards either  gains or losses when populations are small.  Set to true based on previously published data.

\item[\code{Steps}] The number of distance steps away from a target location that are considered "local."  Includes all 8 cardinal and ordinal directions around a target, so the local area is always a rectangle around the target location.

\item[\code{HorizontalLocal}] Whether horizonta transfer occurs between local populations or globally.  Set to FALSE as a control, as global horizontal transfer should abolish local patterns.

\item[\code{NumberRandomHorizontal}] The number of locatiosn to conpare when HorizontalLocal==FALSE.  Should be 8 when Steps==1, 24 when steps==2, 48 when Steps=3, ect.

\item[\code{UpRoot}] Whether establish populations can move (TRUE) or they remain in place for the entire simulation (FALSE).

\item[\code{Death}] Whether a population can die out.

\item[\code{Bering}] Whether to employ barriers that mimick the Bering Strait and Americas.

\item[\code{MigrationSimSteps}] The number of time steps to run each wave of migration.

\item[\code{HorizontalSimSteps}] The number of time steps to spend on horizontal transfer.

\item[\code{Waves}] Whether migration occurs in waves or all seed populations are added at the same time.  If TRUE, there is one wave for each seed population.

\item[\code{Seed}] Sets a seed for reproducibility if an integer instead of NA.

\item[\code{PhonemeDistribution}] The 1) min, 2) mode, and 3) max number of phonemes a population can have when sampling for seed population sizes and when preventing languages from gaining or losing too many phonemes. Default based on real phoneme data.
\end{ldescription}
\end{Arguments}
\inputencoding{utf8}
\HeaderA{GeneratePhonemeProbabilities}{Generate Phoneme Probabilities}{GeneratePhonemeProbabilities}
\keyword{Phonemes}{GeneratePhonemeProbabilities}
%
\begin{Description}\relax
Genetrate a vector of the probability to know each phoneme.
\end{Description}
%
\begin{Usage}
\begin{verbatim}
GeneratePhonemeProbabilities(P)
\end{verbatim}
\end{Usage}
%
\begin{Arguments}
\begin{ldescription}
\item[\code{P}] A list of parameters.
\end{ldescription}
\end{Arguments}
\inputencoding{utf8}
\HeaderA{GenerateSeedLanguage}{Generate Seed Language}{GenerateSeedLanguage}
\keyword{Phonemes}{GenerateSeedLanguage}
%
\begin{Description}\relax
Generate Seed Language
\end{Description}
%
\begin{Usage}
\begin{verbatim}
GenerateSeedLanguage(P, phonemeProbab, seedNum)
\end{verbatim}
\end{Usage}
%
\begin{Arguments}
\begin{ldescription}
\item[\code{P}] A list of parameters.

\item[\code{phonemeProbab}] The probability of gaining each phoneme in the population.

\item[\code{seedNum}] Which population seed is having it's language generated.
\end{ldescription}
\end{Arguments}
\inputencoding{utf8}
\HeaderA{GetFactorDim}{Get Factor Dimentions}{GetFactorDim}
\keyword{Directions}{GetFactorDim}
%
\begin{Description}\relax
Given a number of consomants or vowel, create a datastructure that is as square as possible.
\end{Description}
%
\begin{Usage}
\begin{verbatim}
GetFactorDim(nPhonemes)
\end{verbatim}
\end{Usage}
%
\begin{Arguments}
\begin{ldescription}
\item[\code{nPhonemes}] The number of Phonemes (vowels or consonants).
\end{ldescription}
\end{Arguments}
\inputencoding{utf8}
\HeaderA{GetRealPhonemeData}{Get Real Phoneme Data}{GetRealPhonemeData}
\keyword{Phonemes}{GetRealPhonemeData}
%
\begin{Description}\relax
Uses the real Phoneme data from Creanza..... UPDATE THIS!!!! to determine the phoneme probabilities.
\end{Description}
%
\begin{Usage}
\begin{verbatim}
GetRealPhonemeData(nPhoneme, actual, file)
\end{verbatim}
\end{Usage}
%
\begin{Arguments}
\begin{ldescription}
\item[\code{nPhoneme}] The number of phonemes (vowels or consonants).

\item[\code{actual}] Whether the data is Real (TRUE) or RealMimic (FALSE).
\end{ldescription}
\end{Arguments}
\inputencoding{utf8}
\HeaderA{HCBSimmulation}{Human CUltural Boundaries Simulation}{HCBSimmulation}
\keyword{SimParam}{HCBSimmulation}
%
\begin{Description}\relax
Runs a simulation
\end{Description}
%
\begin{Usage}
\begin{verbatim}
HCBSimmulation(P)
\end{verbatim}
\end{Usage}
%
\begin{Arguments}
\begin{ldescription}
\item[\code{P}] A list of parameters.
\end{ldescription}
\end{Arguments}
\inputencoding{utf8}
\HeaderA{HoritontalTransferRepeater}{After migration, allow populations to exchnage phoneme information, losing or gaining syllables based on other populations in the simulation.}{HoritontalTransferRepeater}
\keyword{Horizontal}{HoritontalTransferRepeater}
%
\begin{Description}\relax
After migration, allow populations to exchnage phoneme information, losing or gaining syllables based on other populations in the simulation.
\end{Description}
%
\begin{Usage}
\begin{verbatim}
HoritontalTransferRepeater(P, S)
\end{verbatim}
\end{Usage}
%
\begin{Arguments}
\begin{ldescription}
\item[\code{P}] A list of parameters.

\item[\code{S}] A list of the data structures.
\end{ldescription}
\end{Arguments}
\inputencoding{utf8}
\HeaderA{HorizontalTransfer}{Horizontal Transfer}{HorizontalTransfer}
\keyword{Horizontal}{HorizontalTransfer}
%
\begin{Description}\relax
A function wrapper that get the language to modify and allows the phoneme change to either add/shift or remove a phoneme if this can be done.
\end{Description}
%
\begin{Usage}
\begin{verbatim}
HorizontalTransfer(P, languages, local, phonemeRelatedness, phonemeProbab,
  index)
\end{verbatim}
\end{Usage}
%
\begin{Arguments}
\begin{ldescription}
\item[\code{P}] A list of parameters.

\item[\code{languages}] All languages.

\item[\code{local}] Locality.

\item[\code{phonemeRelatedness}] The phoneme relatedness list.

\item[\code{phonemeProbab}] The probability of gaining each phoneme in the population.

\item[\code{index}] The target territory whose language may change.
\end{ldescription}
\end{Arguments}
\inputencoding{utf8}
\HeaderA{Initialize}{Initialize}{Initialize}
\keyword{Wrapper}{Initialize}
%
\begin{Description}\relax
The function wrapper that makes calls to create the population and phoneme data structures and then populatas them with initial data.
\end{Description}
%
\begin{Usage}
\begin{verbatim}
Initialize(P)
\end{verbatim}
\end{Usage}
%
\begin{Arguments}
\begin{ldescription}
\item[\code{P}] A list of parameters.
\end{ldescription}
\end{Arguments}
\inputencoding{utf8}
\HeaderA{Lose}{Lose Phoneme}{Lose}
\keyword{Horizontal}{Lose}
%
\begin{Description}\relax
Allows a language to either lose a phoneme to better match other populations.
\end{Description}
%
\begin{Usage}
\begin{verbatim}
Lose(P, language, phonemeProbab)
\end{verbatim}
\end{Usage}
%
\begin{Arguments}
\begin{ldescription}
\item[\code{P}] A list of parameters.

\item[\code{language}] The target language to be modified if possible.

\item[\code{phonemeProbab}] The probability of gaining each phoneme in the population.
\end{ldescription}
\end{Arguments}
\inputencoding{utf8}
\HeaderA{NextStepDirections}{Next Step Directions Expands the Steps list one more step out.}{NextStepDirections}
\keyword{Directions}{NextStepDirections}
%
\begin{Description}\relax
Next Step Directions
Expands the Steps list one more step out.
\end{Description}
%
\begin{Usage}
\begin{verbatim}
NextStepDirections(firstStep, currentStep, start = 0)
\end{verbatim}
\end{Usage}
%
\begin{Arguments}
\begin{ldescription}
\item[\code{firstStep}] The original StepOne.

\item[\code{currentStep}] StepOne in its current state.

\item[\code{start}] How much to offset numbers (for phoneme structures).
\end{ldescription}
\end{Arguments}
\inputencoding{utf8}
\HeaderA{OneStepDirections}{One Step Directions}{OneStepDirections}
\keyword{Directions}{OneStepDirections}
%
\begin{Description}\relax
One Step Directions
\end{Description}
%
\begin{Usage}
\begin{verbatim}
OneStepDirections(R, C, start = 0, round = FALSE)
\end{verbatim}
\end{Usage}
%
\begin{Arguments}
\begin{ldescription}
\item[\code{R}] The number of rows.

\item[\code{C}] The number of columns.

\item[\code{start}] How much to offset numbers (for phoneme structures).

\item[\code{round}] whether to make the spacing Round (Phonemes) or Sqaure (Territories).
\end{ldescription}
\end{Arguments}
\inputencoding{utf8}
\HeaderA{RemoveHorizontalConnections}{Remove Horizontal Connections}{RemoveHorizontalConnections}
\keyword{Barriers}{RemoveHorizontalConnections}
%
\begin{Description}\relax
Affects local territories below/South (and perhaps to the Southeast and Southwest) the target territory (index) and above/North (perhaps Northwest and Northeast) of index +1.
\end{Description}
%
\begin{Usage}
\begin{verbatim}
RemoveHorizontalConnections(R, index, firstStep, right = TRUE,
  left = TRUE)
\end{verbatim}
\end{Usage}
%
\begin{Arguments}
\begin{ldescription}
\item[\code{R}] The number of rows in the population matrix.

\item[\code{index}] The target territory.

\item[\code{firstStep}] The local directions created by OneStepDirections().

\item[\code{right}] Whether to remove the right diagonal.

\item[\code{left}] Whether to remove the left diagonal.
\end{ldescription}
\end{Arguments}
\inputencoding{utf8}
\HeaderA{RemoveVerticalConnections}{Remove Vertical Connections}{RemoveVerticalConnections}
\keyword{Barriers}{RemoveVerticalConnections}
%
\begin{Description}\relax
Affects local territories right/East (and perhaps to the Northeast and Southeast) the target territory (index) and left/West (perhaps Northwest and Southwest) of index + R.
\end{Description}
%
\begin{Usage}
\begin{verbatim}
RemoveVerticalConnections(R, index, firstStep, above = TRUE,
  below = TRUE)
\end{verbatim}
\end{Usage}
%
\begin{Arguments}
\begin{ldescription}
\item[\code{R}] The number of rows in the population matrix.

\item[\code{index}] The target territory.

\item[\code{firstStep}] The local directions created by OneStepDirections().

\item[\code{above}] Whether to remove the upper diagonal.

\item[\code{below}] Whether to remove the lower diagonal.
\end{ldescription}
\end{Arguments}
\inputencoding{utf8}
\HeaderA{ShiftDirections}{Shift Directions}{ShiftDirections}
\keyword{Directions}{ShiftDirections}
%
\begin{Description}\relax
Returns the relationships between phonemes with an offset of start.
\end{Description}
%
\begin{Usage}
\begin{verbatim}
ShiftDirections(nPhonemes, start = 0)
\end{verbatim}
\end{Usage}
%
\begin{Arguments}
\begin{ldescription}
\item[\code{nPhonemes}] The number of phonemes.

\item[\code{start}] Where to start number the phonemes (0 for consonants, number of consonants +1 for vowels).
\end{ldescription}
\end{Arguments}
\inputencoding{utf8}
\HeaderA{StepDirections}{Step Directions}{StepDirections}
\keyword{Directions}{StepDirections}
\keyword{Wrapper}{StepDirections}
%
\begin{Description}\relax
A wrapper that calls StepOne(), add barriers if required, then expands StepOne as many steps as the Steps parameter calls for.
\end{Description}
%
\begin{Usage}
\begin{verbatim}
StepDirections(P)
\end{verbatim}
\end{Usage}
%
\begin{Arguments}
\begin{ldescription}
\item[\code{P}] A list of parameters.
\end{ldescription}
\end{Arguments}
\printindex{}
\end{document}
